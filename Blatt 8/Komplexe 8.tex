\documentclass[a4paper,10pt]{article}
%\documentclass[a4paper,10pt]{scrartcl}

\usepackage{../mystyle}

\setromanfont[Mapping=tex-text]{Linux Libertine O}
% \setsansfont[Mapping=tex-text]{DejaVu Sans}
% \setmonofont[Mapping=tex-text]{DejaVu Sans Mono}

\title{Einführung in die Komplexe Analysis \\ \Large Blatt 8}
\author{Jendrik Stelzner}
\date{\today}

\begin{document}
\maketitle





\section{}


\subsection{}
Es ist
\begin{align*}
 \int_\gamma \Re\left( z^n \right) \dd{z}
 &= \int_0^1 \Re\left( r^n e^{2\pi i n t} \right) 2\pi i r e^{2\pi i t} \dd{t} \\
 &= 2\pi i r^{n+1} \int_0^1 \Re\left( e^{2\pi i n t} \right) e^{2\pi i t} \dd{t} \\
 &= 2\pi i r^{n+1} \int_0^1 \cos( 2\pi n t ) e^{2\pi i t} \dd{t},
\end{align*}
wobei
\begin{align*}
 \int_0^1 \cos( 2\pi n t ) e^{2\pi i t} \dd{t}
 &= \frac{1}{2} \int_0^1 \left( e^{2\pi i n t} + e^{-2\pi i n t} \right) e^{2\pi i t} \dd{t} \\
 &= \frac{1}{2} \int_0^1 e^{2\pi i (1+n) t} + e^{2\pi i (1-n) t} \dd{t}.
\end{align*}
Daher ist
\[
 \int_\gamma \Re\left( z^n \right) \dd{z}
 = \begin{cases} \pi i r^2 & \text{falls } n = 1, \\ \pi i & \text{falls } n = -1, \\ 0 & \text{sonst}. \end{cases}
\]

Analog ergibt sich, dass
\begin{align*}
\int_\gamma \Im\left( z^n \right) \dd{z}
 &= 2\pi i r^{n+1} \int_0^1 \sin( 2\pi n t ) e^{2\pi i t} \dd{t} \\
 &= \pi r^{n+1} \int_0^1 e^{2\pi i (1+n) t} - e^{2\pi i (1-n) t} \dd{t} \\
 &=
 \begin{cases}
  -\pi r^2 & \text{falls } n=1, \\
   \pi     & \text{falls } n = -1, \\
         0 & \text{sonst}.
 \end{cases}
\end{align*}


\subsection{}
Es ist
\begin{align*}
 \int_\gamma \overline{z}^n \dd{z}
 &= \int_0^1 \overline{r e^{2\pi i t}}^n 2\pi i r e^{2\pi i t} \dd{t}
 = 2\pi i r^{n+1} \int_0^1 e^{2\pi i (1-n) t} \dd{t} \\
 &= \begin{cases} 2\pi i r^2 & \text{falls } n=1, \\ 0 & \text{sonst}. \end{cases}
\end{align*}


\subsection{}
Ich zeige nur eine schwächere Aussage: Es sei $r > 0$ klein genug, so dass $B_r(z_0) \subseteq U$ und sich $f$ auf $B_r(z_0)$ als Potenzreihe
\[
 f(z) = \sum_{n=0}^\infty a_n (z-z_0)^n
\]
darstellen lässt. Dann ist auch
\[
 f'(z) = \sum_{n=1}^\infty n a_n (z-z_0)^{n-1} \text{ für alle } z \in B_r(z_0),
\]
und insbesondere $f'(z_0) = a_1$. Definieren wir für $0 <\rho < r$
\[
 \gamma_\rho : [0,1] \to U \text{ als } \gamma(t) := z_0 + \rho e^{2\pi i t} \text{ für alle } t \in [0,1],
\]
so ist
\begin{align*}
 \oint_{|z-z_0| = \rho} \overline{f(z)} \dd{z}
 &= \int_0^1 \overline{f(z_0 + \rho e^{2\pi i t})} \cdot 2 \pi i \rho e^{2\pi it} \dd{t} \\
 &= \int_0^1 \overline{\sum_{n=0}^\infty a_n \left(\rho e^{2\pi i t}\right)^n} \cdot 2 \pi i \rho e^{2\pi i t} \\
 &= \sum_{n=0}^\infty \overline{a_n} \int_0^1 \overline{\rho e^{2\pi i t}}^n \cdot 2 \pi i \rho e^{2\pi i t} \\
 &= \sum_{n=0}^\infty \overline{a_n} \int_{\gamma_\rho} \overline{z}^n \dd{z}
 = \overline{a_1} 2 \pi i \rho^2
 = 2 \pi i \rho^2 \overline{f'(z_0)}.
\end{align*}
Dabei haben wir genutzt, dass die Potenzreihe auf $B_\rho(z_0)$ gleichmäßig konvergiert, und Summe und Integral deshalb vertauscht werden dürfen. Die Werte der einzelnen Wegintegrale ergeben sich aus dem vorherigen Aufgabenteil.








\addtocounter{section}{1}





\section{}
Da $[a,b]$ kompakt ist, und $f \circ \gamma : [a,b] \to \C$ stetig ist, existiert das Maximum
\[
 M := \max_{a \leq t \leq b} |f(\gamma(t))|,
\]
und es gilt
\begin{align*}
 \left| \int_\gamma f(z) \dd{z} \right|
 &= \left| \int_a^b f(\gamma(t)) \gamma'(t) \dd{t} \right|
 \leq \int_a^b |f(\gamma(t) \gamma'(t) | \dd{t} \\
 &\leq M \int_a^b |\gamma'(t)| \dd{t}
 = M L(\gamma).
\end{align*}





\section{}
Wir fixieren zunächst einen Basispunkt $a \in U$. Wegen der Konvexität von $U$ existieren für jedes $z \in \C$ der Weg
\[
 \gamma_z : [0,1] \to U \text{ mit } \gamma_z(t) := a + t(z-a) \text{ für alle } t \in [0,1].
\]
Wir definieren $F : U \to \C$ durch
\[
 F(z) := \int_{\gamma_z} f(z) \dd{z} \text{ für alle } z \in U,
\]
und behaupten, dass $F$ auf $U$ eine Stammfunktion von $f$ ist.

Sei $z_0 \in U$ beliebig aber fest. Wir zeigen, dass $F$ an $z_0$ komplex differenzierbar ist mit $F'(z_0) = f(z_0)$. Hierfür definieren wir für alle $z \in \C$ den Weg
\[
 \Gamma_z : [0,1] \to U \text{ mit } \Gamma_z(t) := z_0 + t(z-z_0) \text{ für alle } t \in [0,1],
\]
der wegen der Konvexität von $U$ wohldefiniert ist. Für alle $z \in U$ ist dann
\begin{align*}
 0
 &= \int_{\gamma_{z_0} + \Gamma_z - \gamma_z} f(z) \dd{z}
 = \int_{\gamma_{z_0}} f(z) \dd{z} - \int_{\gamma_z} f(z) \dd{z} + \int_{\Gamma_z} f(z) \dd{z} \\
 &= F(z_0) - F(z) + \int_0^1 f(z_0 + t(z-z_0))(z-z_0) \dd{t},
\end{align*}
also
\[
 F(z) = F(z_0) + (z-z_0) \Delta(z),
\]
wobei $\Delta : U \to \C$ als
\[
 \Delta(z) := \int_0^1 f(z_0 + t(z-z_0)) \dd{t} \text{ für alle } z \in U
\]
definiert ist.

Es ist klar, dass $\Delta(z_0) = f(z_0)$. $\Delta$ ist stetig an $z_0$, denn $f$ ist stetig, und für alle $z \in U$ ist
\begin{align*}
 |\Delta(z)-\Delta(z_0)|
 &= \left| \int_0^1 f(z_0+t(z-z_0)) - f(z_0) \dd{t} \right| \\
 &\leq \max_{0 \leq t \leq 1} |f(z_0+t(z-z_0))-f(z_0)|.
\end{align*}
Zusammen zeigt, dies, dass $F$ an $z_0$ komplex differenzierbar ist mit $F'(z_0) = f(z)$. Aus der Beliebigkeit von $z_0 \in U$ folgt, dass $F$ auf $U$ holomorph ist mit $F' = f$. Also besitzt $f$ auf $U$ eine Stammfunktion.

Sei $U \subseteq \C$ nichtleer, offen und konvex. Für $z_1, z_2 \in U$ mit $z_1 = x_1 + iy_1$ und $z_2 = x_2 + iy_2$ und den Weg
\[
 \gamma : [0,1] \to U, z_1 + t(z_2 - z_1)
\]
ist
\begin{align*}
 \int_\gamma \Re(z) \dd{z}
 &= \int_0^1 \Re(z_1 + t(z_2 - z_1)) (z_2 - z_1) \dd{t} \\
 &= (z_2 - z_1) \int_0^1 x_1 + t(x_2 - x_1) \dd{t} \\
 &= (z_2 - z_1) \left( x_1 + \frac{x_2 - x_1}{2} \right) \\
 &= (z_2 - z_1) \frac{x_1 + x_2}{2}.
\end{align*}

Seien $x+iy, x'+iy' \in U$ mit $x \neq x'$ und $y \neq y'$, so dass $x'+iy \in U$. Für das von $x+iy$, $x'+iy'$ und $x+iy'$ aufgespannte Dreieck $\Delta$ ergibt sich dann
\begin{align*}
  &\, \int_{\partial \Delta} f(z) \dd{z} \\
 =&\, (x'-x+i(y'-y))\frac{x+x'}{2} + (x-x')\frac{x+x'}{2} + i(y-y')\frac{x+x}{2} \\
 =&\, i(y'-y) \frac{x+x'}{2} - i(y'-y)x
 = \frac{i}{2}(y'-y)(x'-x)
 \neq 0.
\end{align*}
Deshalb kann $\Re$ auf $U$ kein Stammfunktion besitzen. Da jede nichtleere offene Menge $U \subseteq \C$ eine nichtleere, offene, konvexe Teilmenge besitzt, besitzt $\Re$ auf keiner nichtleeren offenen Mengen eine Stammfunktion. Daher besitzt auch $\Im$ auf keiner nichtleeren offenen Menge eine Stammfunktion, da wegen $\Re(z) = \Im(-iz)$ sonst auch $\Re$ dort eine Stammfunktion hätte.







\end{document}
