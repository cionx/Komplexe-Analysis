\documentclass[a4paper,10pt]{article}
%\documentclass[a4paper,10pt]{scrartcl}

\usepackage{../mystyle}

\setromanfont[Mapping=tex-text]{Linux Libertine O}
% \setsansfont[Mapping=tex-text]{DejaVu Sans}
% \setmonofont[Mapping=tex-text]{DejaVu Sans Mono}

\title{\sc Einführung in die Komplexe Analysis \\ \Large Blatt 6}
\author{Jendrik Stelzner}
\date{\today}

\begin{document}
\maketitle





\section{(Konvergenzradius)}
Da
\[
 \sum_n a_n z^{2n}
 = \sum_n a_n \left(z^2\right)^n
\]
beträgt der Konvergenzradius der Reihe $\sum_n a_n z^{2n}$
\[
 \sqrt{R} \text{ falls } R < \infty \quad \text{und} \quad \infty \text{ falls } R = \infty.
\]
Für die Reihe $\sum_n a_n^2 z^n$ ergibt sich der Konvergenzradius
\[
 \frac{1}{\limsup_{n \to \infty} \sqrt[n]{|a_n^2|}}
 = \left( \frac{1}{\limsup_{n \to \infty} \sqrt{|a_n|}} \right)^2
 = R^2,
\]
wobei wir $\infty^2 = \infty$ verstehen. Kombiniert ergibt sich damit, dass $\sum_n a_n^2 z^{2n}$ einen Konvergenzradius von $R$ hat.

Da
\[
 0 < R = \frac{1}{\limsup_{n \to \infty} \sqrt[n]{|a_n|}}
\]
ist
\[
 \limsup_{n \to \infty} \sqrt[n]{|a_n|} < \infty.
\]
Deshalb ist
\[
 \frac{1}{\limsup_{n \to \infty} \sqrt[n]{|a_n|/n!}}
 = \frac{\lim_{n \to \infty} \sqrt[n]{n!}}{\limsup_{n \to \infty} \sqrt[n]{|a_n|}}
 = \infty.
\]
Der Konvergenzradius von $\sum_n (a_n/n!) z^n$ ist daher $\infty$.





\section{(Abelscher Grenzwertsatz)}
Wir betrachten die Potenzreihe $\sum_{n=0}^\infty a_n (z-1)^n$ von $f = \log$ an der Stelle $x_0 := 1$. Induktiv ergibt sich, dass
\[
 f^{(n)}(x) = (-1)^{n-1} \frac{(n-1)!}{x^n} \text{ für alle } n \geq 1 \text{ und } x > 0.
\]
Daher ist
\[
 f(x_0) = 0 \text{ und } f^{(n)}(x_0) = (-1)^{n-1}(n-1)! \text{ für alle } n \geq 1.
\]
Die Koeffizienten $(a_n)_{n \in \N}$ der Potenzreihe sind daher
\[
 a_0 = f(x_0) = 0 \text{ und } a_n = \frac{f^{(n)}(x_0)}{n!} = \frac{(-1)^{n-1}}{n} \text{ für alle } n \geq 1.
\]
Da
\[
 \limsup_{n \to \infty} |a_n|^{1/n} = \limsup_{n \to \infty} \frac{1}{n^{1/n}} = 1
\]
beträgt der Konvergenzradius der Reihe $1/1 = 1$. Für alle $z \in \C$ mit $|z-1| < 1$ konvergiert daher die Reihe $\sum_{n=0}^\infty a_n (z-1)^n$ (dies folgt auch direkt daraus, dass $(a_n)$ eine Nullfolge ist) und es ist
\[
 \sum_{n=0}^\infty a_n (x-1)^n = \log(x) \text{ für alle } 0 < x < 2.
\]
Bekanntermaßen konvergiert die Reihe $\sum_{n=0}^\infty a_n = \sum_{n=1}^\infty (-1)^{n-1}/n$ nach dem Leibniz-Kriterium (aus Analysis 1 ist diese Reihe auch als alternierende harmonische Reihe bekannt). Nach dem Abelschen Grenzwertsatz ist daher
\[
 \log(2)
 = \lim_{x \uparrow 2} \log(x)
 = \lim_{x \uparrow 2} \sum_{n=0}^\infty a_n (x-1)^n
 = \sum_{n=0}^\infty a_n
 = \sum_{n=1}^\infty \frac{(-1)^{n-1}}{n},
\]
also
\[
 \sum_{n=1}^\infty \frac{(-1)^n}{n} = -\log(2).
\]





\section{(Bessel-Funktion)}
Für alle $z \in \C$ ist
\[
 J(z)
 = \sum_{n=0}^\infty \frac{(-1)^n}{(n!)^2} \left( \frac{z}{2} \right)^n
 = \sum_{n=0}^\infty \frac{(-1)^n}{(n!)^2 4^n} z^{2n}.
\]

Da
\[
 \limsup_{n \to \infty} \left( \frac{1}{(n!)^2 4^n} \right) = \limsup_{n \to \infty} \frac{1}{4 (n!)^{2/n}} = 0
\]
hat die Potenzreihe einen Konvergenzradius von $\infty$. Wir können $J$ daher summandenweise ableiten. Deshalb ist
\begin{align*}
  J'(z) &= \sum_{n=1}^\infty \frac{(-1)^n 2n}{(n!)^2 4^n}{z^{2n-1}} \text{ und} \\
 J''(z) &= \sum_{n=1}^\infty \frac{(-1)^n 2n (2n-1)}{(n!)^2 4^n} z^{2n-2}.
\end{align*}
Daher ist für alle $z \in \C$
\begin{align*}
 z^2 J''(z) + z J'(z)
 &= \sum_{n=1}^\infty (-1)^n \frac{2n(2n-1)}{(n!)^2 4^n} z^{2n} + \sum_{n=1}^\infty (-1)^n \frac{2n}{(n!)^2 4^n} z^{2n} \\
 &= \sum_{n=1}^\infty (-1)^n \frac{2n(2n-1)}{(n!)4^n} z^{2n} + (-1)^n \frac{2n}{(n!)^2 4^n} z^{2n} \\
 &= \sum_{n=1}^\infty (-1)^n \frac{4n^2}{(n!)^2 4^n} \\
 &= \sum_{n=1}^\infty (-1)^n \frac{1}{((n-1)!)^2 4^{n-1}} z^{2n} \\
 &= \sum_{n=1}^\infty (-1)^{n+1} \frac{1}{(n!)^2 4^n} z^{2n+2} \\
 &= -z^2 \sum_{n=0}^\infty \frac{(-1)^n}{(n!)^2 4^n} z^{2n} \\
 &= -z^2 J(z).
\end{align*}





\section{(Konstruktion von Stammfunktionen)}
Für alle $z \in \C$ ist
\begin{align*}
 F(z)
 &= \int_{\gamma_z} \xi e^{\xi} \dd{\xi}
 = \int_0^1 tz e^{tz} z \dd{t}
 = \int_0^1 t z^2 e^{tz} \dd{t} \\
 &= \left[ (tz-1) e^{tz} \right]_{t=0}^1
 = (z-1)e^z + 1.
\end{align*}
$F$ ist offenbar auf ganz $\C$ holomorph.

Für alle $z \in \C$ ist
\begin{align*}
 G(z)
 &= \int_{\gamma_z} |\xi|^2 \dd{\xi}
 = \int_0^1 |tz|^2 z \dd{t}
 = |z|^2 z \int_0^1 t^2 \dd{t}
 = \frac{1}{3} |z|^2 z
 = \frac{1}{3} z^2 \overline{z}.
\end{align*}
$G$ ist an $z = 0$ komplex differenzierbar, da
\[
 \lim_{h \to 0} \frac{G(h)-G(0)}{h}
 = \lim_{h \to 0} \frac{|h|^2 h}{3h}
 = \lim_{h \to 0} \frac{1}{3} |h|^2
 = 0.
\]
Für $z \neq 0$ ist $G$ nicht komplex differenzierbar an $z$, denn ansonsten wäre nach der Quotientenregel auch
\[
 \overline{z} = \frac{G(z)}{(1/3)z^2}
\]
komplex differenzierbar an $z$, was aber nicht gilt.





\section{(Interpretation des komplexen Kurvenintegrals)}


\subsection{}
Wir bemerken zunächst, dass die Parametrisierung über $\Img \gamma$, d.h. $\nu : \Img \gamma \to \C$, problematisch ist, da $\gamma$ nicht notwendigerweise injektiv ist, also $\Img \gamma$ Selbstschnitte haben kann. Wir parametrisieren daher $\nu$ über $[a,b]$.

Damit $\nu$ eine Normale ist, muss
\[
 \scal{\nu(t), \gamma'(t)} = 0 \text{ für alle } t \in [a,b].
\]
Daher muss
\[
 \nu(t) \in \left( \R \gamma'(t) \right)^{\bot} = i \R \gamma'(t) \text{ für alle } t \in [a,b].
\]
(Man beachte, dass $\gamma'(t) \neq 0$ für alle $t \in [a,b]$, und dass Multiplikation mit $i$ der Rotation um $\pi/2$ entspricht.) Da $\nu$ auch normiert ist, muss
\[
 \nu(t) = \pm i \frac{\gamma'(t)}{|\gamma'(t)|} \text{ für alle } t \in [a,b].
\]
Da
\[
 |\gamma'(t)| = |J_\gamma(t)^T J_\gamma(t)| \text{ für alle } t \in [a,b]
\]
zeigt dies die Aussage.


\subsection{}
Wir bemerken zunächst, dass
\[
 \mf{v}_f = \vect{\Re(f) \\ \Im(f)}
\]
gewählt werden muss, damit die Aussage gilt. Denn schreiben wir
\[
 \gamma_1 := \Re(\gamma), \gamma_2 = \Im(\gamma), u = \Re(f) \text{ und } v = \Im(f),
\]
so ist
\begin{align*}
 &\, \int_\gamma f(z) \dd{z}
 = \int_a^b f(\gamma(t)) \gamma'(t) \dd{t} \\
 =&\, \int_a^b u(\gamma(t))\gamma_1'(t) - v(\gamma(t))\gamma_2'(t) \dd{t}
   + i \int_a^b u(\gamma(t)) \gamma_2'(t) + v(\gamma(t)) \gamma_1'(t) \dd{t}.
\end{align*}
Da für alle $t \in [a,b]$
\begin{align*}
 \nu(t) = -i\left|J_\gamma(t)^T J_\gamma(t)\right|^{-1/2} \gamma'(t) = |\gamma'(t)|^{-1/2} (\gamma_2'(t) - i\gamma_1'(t))
\end{align*}
ist
\begin{align*}
 \int_\gamma \mf{v}_{\overline{f}} \dd{x}
 &= \int_a^b \scal{\mf{v}_{\overline{f}}(\gamma(t)), \gamma'(t)} \dd{t} \\
 &= \int_a^b u(\gamma(t))\gamma_1'(t) - v(\gamma(t))\gamma_2'(t) \dd{t}
\end{align*}
und
\begin{align*}
 \int_\gamma \mf{v}_{\overline{f}} \dd{\vec{\sigma}}
 &= \int_a^b \scal{\mf{v}_{\overline{f}}(\gamma(t)), \nu(t)} \left|J_\gamma(t)^T J_\gamma(t)\right|^{1/2} \dd{t} \\
 &= \int_a^b u(\gamma(t))\gamma_2'(t) + v(\gamma(t))\gamma_1'(t) \dd{t}.
\end{align*}
Das zeigt die Gleichheit.


\subsection{}
Es ist
\[
 \diver( \mf{v}_{\overline{f}} )
 = u_x + (-v)_y
 = u_x - v_y
 = 0,
\]
denn aus der Holomorphie von $f$ folgt, dass $f$ die Cauchy-Riemannschen Differentialgleichungen
\[
 u_x = v_y \quad \text{und} \quad u_y = -v_x
\]
erfüllt.


\subsection{}
Nach dem Satz von Gauß ist in der gegeben Situation
\[
 \Im\left( \int_\gamma f(z) \dd{z} \right)
 = \int_\gamma \mf{v}_{\overline{f}} \dd{\vec{\sigma}}
 = \int_\Omega \diver(\mf{v}_{\overline{f}}) \dd{\lambda_2}
 = \int_\Omega 0 \dd{\lambda_2}
 = 0.
\]
Da $f$ holomorph ist, ist auch $if$ holomorph. Daher ist nach analoger Argumentation
\[
 \Re\left( \int_\gamma f(z) \dd{z} \right)
 = \Im\left( i \int_\gamma f(z) \dd{z} \right)
 = \Im\left( \int_\gamma if(z) \dd{z} \right)
 = 0.
\]
Also ist
\[
 \int_\gamma f(z) \dd{z} = 0.
\]






























\end{document}
