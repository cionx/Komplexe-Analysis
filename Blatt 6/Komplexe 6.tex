\documentclass[a4paper,10pt]{article}
%\documentclass[a4paper,10pt]{scrartcl}

\usepackage{../mystyle}

\setromanfont[Mapping=tex-text]{Linux Libertine O}
% \setsansfont[Mapping=tex-text]{DejaVu Sans}
% \setmonofont[Mapping=tex-text]{DejaVu Sans Mono}

\title{\sc Einführung in die Komplexe Analysis \\ \Large Blatt 6}
\author{Jendrik Stelzner}
\date{\today}

\begin{document}
\maketitle





\section{(Konvergenzradius)}
Da
\[
 \sum_n a_n z^{2n}
 = \sum_n a_n \left(z^2\right)^n
\]
beträgt der Konvergenzradius der Reihe $\sum_n a_n z^{2n}$
\[
 \sqrt{R} \text{ falls } R < \infty \quad \text{und} \quad \infty \text{ falls } R = \infty.
\]
Für die Reihe $\sum_n a_n^2 z^n$ ergibt sich der Konvergenzradius
\[
 \frac{1}{\limsup_{n \to \infty} \sqrt[n]{|a_n^2|}}
 = \left( \frac{1}{\limsup_{n \to \infty} \sqrt{|a_n|}} \right)^2
 = R^2,
\]
wobei wir $\infty^2 = \infty$ verstehen. Kombiniert ergibt sich damit, dass $\sum_n a_n^2 z^{2n}$ einen Konvergenzradius von $R$ hat.

Da
\[
 0 < R = \frac{1}{\limsup_{n \to \infty} \sqrt[n]{|a_n|}}
\]
ist
\[
 \limsup_{n \to \infty} \sqrt[n]{|a_n|} < \infty.
\]
Deshalb ist
\[
 \frac{1}{\limsup_{n \to \infty} \sqrt[n]{|a_n|/n!}}
 = \frac{\lim_{n \to \infty} \sqrt[n]{n!}}{\limsup_{n \to \infty} \sqrt[n]{|a_n|}}
 = \infty.
\]
Der Konvergenzradius von $\sum_n (a_n/n!) z^n$ ist daher $\infty$.





\section{(Abelscher Grenzwertsatz)}
Wir betrachten die Potenzreihe $\sum_{n=0}^\infty a_n (z-1)^n$ von $f = \log$ an der Stelle $x_0 := 1$. Induktiv ergibt sich, dass
\[
 f^{(n)}(x) = (-1)^{n-1} \frac{(n-1)!}{x^n} \text{ für alle } n \geq 1 \text{ und } x > 0.
\]
Daher ist
\[
 f(x_0) = 0 \text{ und } f^{(n)}(x_0) = (-1)^{n-1}(n-1)! \text{ für alle } n \geq 1.
\]
Die Koeffizienten $(a_n)_{n \in \N}$ der Potenzreihe sind daher
\[
 a_0 = f(x_0) = 0 \text{ und } a_n = \frac{f^{(n)}(x_0)}{n!} = \frac{(-1)^{n-1}}{n} \text{ für alle } n \geq 1.
\]
Da
\[
 \limsup_{n \to \infty} |a_n|^{1/n} = \limsup_{n \to \infty} \frac{1}{n^{1/n}} = 1
\]
beträgt der Konvergenzradius der Reihe $1/1 = 1$. Für alle $z \in \C$ mit $|z-1| < 1$ konvergiert daher die Reihe $\sum_{n=0}^\infty a_n (z-1)^n$ (dies folgt auch direkt daraus, dass $(a_n)$ eine Nullfolge ist) und es ist
\[
 \sum_{n=0}^\infty a_n (x-1)^n = \log(x) \text{ für alle } 0 < x < 2.
\]
Bekanntermaßen konvergiert die Reihe $\sum_{n=0}^\infty a_n = \sum_{n=1}^\infty (-1)^{n-1}/n$ nach dem Leibniz-Kriterium (aus Analysis 1 ist diese Reihe auch als alternierende harmonische Reihe bekannt). Nach dem Abelschen Grenzwertsatz ist daher
\[
 \log(2)
 = \lim_{x \uparrow 2} \log(x)
 = \lim_{x \uparrow 2} \sum_{n=0}^\infty a_n (x-1)^n
 = \sum_{n=0}^\infty a_n
 = \sum_{n=1}^\infty \frac{(-1)^{n-1}}{n},
\]
also
\[
 \sum_{n=1}^\infty \frac{(-1)^n}{n} = -\log(2).
\]



















\end{document}
