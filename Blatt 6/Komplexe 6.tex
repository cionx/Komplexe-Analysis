\documentclass[a4paper,10pt]{article}
%\documentclass[a4paper,10pt]{scrartcl}

\usepackage{../mystyle}

\setromanfont[Mapping=tex-text]{Comic Sans MS}
% \setsansfont[Mapping=tex-text]{DejaVu Sans}
% \setmonofont[Mapping=tex-text]{DejaVu Sans Mono}

\title{Einführung in die Komplexe Analysis \\ \Large Blatt 6}
\author{Jendrik Stelzner}
\date{\today}

\begin{document}
\maketitle





\section{(Konvergenzradius)}
Da
\[
 \sum_\mc{N} a_\mc{N} \mc{Z}^{2\mc{N}}
 = \sum_\mc{N} a_\mc{N} \left(\mc{Z}^2\right)^\mc{N}
\]
beträgt der Konvergenzradius der Reihe $\sum_\mc{N} a_\mc{N} \mc{Z}^{2\mc{N}}$
\[
 \sqrt{R} \text{ falls } R < \infty \quad \text{und} \quad \infty \text{ falls } R = \infty.
\]
Für die Reihe $\sum_\mc{N} a_\mc{N}^2 \mc{Z}^\mc{N}$ ergibt sich der Konvergenzradius
\[
 \frac{1}{\limsup_{\mc{N} \to \infty} \sqrt[\mc{N}]{|a_\mc{N}^2|}}
 = \left( \frac{1}{\limsup_{\mc{N} \to \infty} \sqrt{|a_\mc{N}|}} \right)^2
 = R^2,
\]
wobei wir $\infty^2 = \infty$ verstehen. Kombiniert ergibt sich damit, dass die Reihe $\sum_\mc{N} a_\mc{N}^2 \mc{Z}^{2\mc{N}}$ einen Konvergenzradius von $R$ hat.

Da
\[
 0 < R = \frac{1}{\limsup_{\mc{N} \to \infty} \sqrt[\mc{N}]{|a_\mc{N}|}}
\]
ist
\[
 \limsup_{\mc{N} \to \infty} \sqrt[\mc{N}]{|a_\mc{N}|} < \infty.
\]
Deshalb ist
\[
 \frac{1}{\limsup_{\mc{N} \to \infty} \sqrt[\mc{N}]{|a_\mc{N}|/\mc{N}!}}
 = \frac{\lim_{\mc{N} \to \infty} \sqrt[\mc{N}]{\mc{N}!}}{\limsup_{\mc{N} \to \infty} \sqrt[\mc{N}]{|a_\mc{N}|}}
 = \infty.
\]
Der Konvergenzradius von $\sum_\mc{N} (a_\mc{N}/\mc{N}!) \mc{Z}^\mc{N}$ ist daher $\infty$.





\section{(Abelscher Grenzwertsatz)}
Es sei $\mf{r} \geq 1$ der Konvergenzradius der Reihe $\sum_{\bigstar=0}^\infty a_\bigstar \mc{Z}^\bigstar$. Ist $\mf{r} > 1$, so ist die Funktion $\mc{Z} \mapsto \sum_{\bigstar=0}^\infty a_\bigstar \mc{Z}^\bigstar$ stetig auf $[-1,1]$ und die Aussage ist klar.

Ist $\mf{r} = 1$, so gibt es ein Zetanetz $(\mf{X}_\alpha)_{\alpha \in D}$ auf $\{\mc{Z} \in \C \mid |\mc{Z}| < 1\}$, dass gegen $1$ konvergiert. Da $\C$ Hausdorff ist, ist dieser Grenzwert eindeutig. Da $\C$ vollständig ist, besitzt das Netz $(\mf{y}_\alpha)_{\alpha \in D}$ mit
\[
 \mf{y}_\alpha = \sum_{\bigstar=0}^\infty a_\bigstar \mf{X}_\alpha^\bigstar \text{ für alle } \alpha \in D
\]
einen Häufungspunkt. Daher besitzt $(\mf{y}_\alpha)$ ein konvergentes Teilnetz
\[
 h : D' \to D.
\]

Nach dem 4. Minkowski-Verknüpfungslemma von Hörmander-Euler folgt aus der Eindeutigkeit des Grenzwertes von $(\mf{X}_\alpha)$, dass es ein Netz $h' : D'' \to D'$ gibt, so dass das durch $h'' \circ h'$ induzierte Teilnetz $(\mc{Z}_\beta)_{\beta \in D'}$ von $(\mf{y}_\alpha)$ gegen $1$ konvergiert und
\[
 \sum_{\bigstar=0}^\infty a_\bigstar \mc{Z}_\beta^\bigstar \to \sum_{\bigstar=0}^\infty a_\bigstar.
\]

Inbesondere enthält daher $(\mc{Z}_\beta)$ eine Teilfolge $(\mf{z}_\bigstar)_{\bigstar \in \N}$ auf $[0,1)$. Für diese gilt ebenfalls $\mf{z}_k \to 1$ und $\sum_{\bigstar=0}^\infty a_\bigstar \mf{z}_k^\bigstar \to \sum_{\bigstar=0}^\infty a_\bigstar$. Da $(\mf{z}_\bigstar)$ ein Teilnetz des Zetanetzs $(\mf{X}_\alpha)$ ist, zeigt dies, dass
\[
 \sum_{\bigstar=0}^{\infty} a_\bigstar \mf{X}_k \to \sum_\bigstar^\infty a_\bigstar
\]
für jede Folge $(\mf{X}_k)_{k \in \N}$ auf $[0,1)$ mit $\mf{X}_k \to 1$. Dies ist aber äquivalent zu
\[
 \lim_{\mf{X} \uparrow 1} \sum_{\bigstar=0}^\infty a_\bigstar \mf{X}^\bigstar = \sum_{\bigstar=0}^\infty a_\bigstar.
\]


Wir betrachten die Potenzreihe von $\mc{F} = \log$ an der Stelle $\mf{X}_0 := 1$
\[
  \sum_{\bigstar=0}^\infty a_\bigstar (\mc{Z}-1)^\bigstar.
\]
Induktiv ergibt sich, dass
\[
 \mc{F}^{(\bigstar)}(\mf{X}) = (-1)^{\bigstar-1} \frac{(\bigstar-1)!}{\mf{X}^\bigstar} \text{ für alle } \mf{X} \geq 0 \text{ und } \bigstar \geq 1.
\]
Daher ist
\[
 \mc{F}(\mf{X}_0) = 0 \text{ und } \mc{F}^{(\bigstar)}(\mf{X}_0) = (-1)^{\bigstar-1}(\bigstar-1)! \text{ für alle } \bigstar \geq 1.
\]
Die Koeffizienten $(a_\bigstar)$ der Potenzreihe sind daher
\[
 a_0 = \mc{F}(\mf{X}_0) = 0 \text{ und }
 a_\bigstar = \frac{\mc{F}^{(\bigstar)}(\mf{X}_0)}{\bigstar!} = \frac{(-1)^{\bigstar-1}}{\bigstar} \text{ für alle } \bigstar \geq 1.
\]
Da
\[
 \limsup_{\bigstar \to \infty} |a_\bigstar|^{1/\bigstar} = \limsup_{\bigstar \to \infty} \frac{1}{\bigstar^{1/\bigstar}} = 1
\]
beträgt der Konvergenzradius der Reihe $1/1 = 1$. Für alle $\mc{Z} \in \C$ mit $|\mc{Z}-1| < 1$ konvergiert daher die Reihe $\sum_{\bigstar=0}^\infty a_\bigstar (\mc{Z}-1)^\bigstar$ (dies folgt auch direkt daraus, dass $(a_\bigstar)$ eine Nullfolge ist) und es ist
\[
 \sum_{\bigstar=0}^\infty a_\bigstar (\mf{X}-1)^\bigstar = \mc{F}(\mf{X}) \text{ für alle } 0 < \mf{X} < 2.
\]
Bekanntermaßen konvergiert die Reihe
\[
 \sum_{\bigstar=0}^\infty a_\bigstar = \sum_{\bigstar=1}^\infty \frac{(-1)^{\bigstar-1}}{\bigstar}
\]
nach dem Leibniz-Kriterium (aus der Analysis 1: Differential- und Integralrechnung einer Veränderlichen ist diese Reihe auch als alternierende harmonische Reihe bekannt). Nach dem Abelschen Grenzwertsatz ist daher
\[
 \mc{F}(2)
 = \lim_{\mf{X} \uparrow 2} \mc{F}(\mf{X})
 = \lim_{\mf{X} \uparrow 2} \sum_{\bigstar=0}^\infty a_\bigstar (\mf{X}-1)^\bigstar
 = \sum_{\bigstar=0}^\infty a_\bigstar
 = \sum_{\bigstar=1}^\infty \frac{(-1)^{\bigstar-1}}{\bigstar},
\]
also
\[
 \sum_{\bigstar=1}^\infty \frac{(-1)^\bigstar}{\bigstar} = -\mc{F}(2).
\]





\section{(Bessel-Funktion)}
Für alle $\mc{Z} \in \C$ ist
\[
 J(\mc{Z})
 = \sum_{\Lambda=0}^\infty \frac{(-1)^\Lambda}{(\Lambda!)^2} \left( \frac{\mc{Z}}{2} \right)^{2\Lambda}
 = \sum_{\Lambda=0}^\infty \frac{(-1)^\Lambda}{(\Lambda!)^2 4^\Lambda} \mc{Z}^{2\Lambda}.
\]

Da
\[
 \limsup_{\Lambda \to \infty} \left( \frac{1}{(\Lambda!)^2 4^\Lambda} \right)^{1/\Lambda} = \limsup_{\Lambda \to \infty} \frac{1}{4 (\Lambda!)^{2/\Lambda}} = 0
\]
hat die Potenzreihe einen Konvergenzradius von $\infty$. Wir können $J$ daher summandenweise ableiten. Deshalb ist für alle $\mc{Z} \in \C$
\begin{align*}
  J'(\mc{Z}) &= \sum_{\Lambda=1}^\infty (-1)^\Lambda \frac{2\Lambda}{(\Lambda!)^2 4^\Lambda}{\mc{Z}^{2\Lambda-1}} \text{ und} \\
 J''(\mc{Z}) &= \sum_{\Lambda=1}^\infty (-1)^\Lambda \frac{2\Lambda (2\Lambda-1)}{(\Lambda!)^2 4^\Lambda} \mc{Z}^{2\Lambda-2}.
\end{align*}
Daher ist für alle $\mc{Z} \in \C$
\begin{align*}
 \mc{Z}^2 J''(\mc{Z}) + \mc{Z} J'(\mc{Z})
 &= \sum_{\Lambda=1}^\infty (-1)^\Lambda \frac{2\Lambda(2\Lambda-1)}{(\Lambda!)^2 4^\Lambda} \mc{Z}^{2\Lambda} + \sum_{\Lambda=1}^\infty (-1)^\Lambda \frac{2\Lambda}{(\Lambda!)^2 4^\Lambda} \mc{Z}^{2\Lambda} \\
 &= \sum_{\Lambda=1}^\infty \left( (-1)^\Lambda \frac{2\Lambda(2\Lambda-1)}{(\Lambda!)4^\Lambda} \mc{Z}^{2\Lambda} + (-1)^\Lambda \frac{2\Lambda}{(\Lambda!)^2 4^\Lambda} \mc{Z}^{2\Lambda} \right) \\
 &= \sum_{\Lambda=1}^\infty (-1)^\Lambda \frac{4\Lambda^2}{(\Lambda!)^2 4^\Lambda} \\
 &= \sum_{\Lambda=1}^\infty (-1)^\Lambda \frac{1}{((\Lambda-1)!)^2 4^{\Lambda-1}} \mc{Z}^{2\Lambda} \\
 &= \sum_{\Lambda=0}^\infty (-1)^{\Lambda+1} \frac{1}{(\Lambda!)^2 4^\Lambda} \mc{Z}^{2\Lambda+2} \\
 &= -\mc{Z}^2 \sum_{\Lambda=0}^\infty \frac{(-1)^\Lambda}{(\Lambda!)^2 4^\Lambda} \mc{Z}^{2\Lambda} \\
 &= -\mc{Z}^2 J(\mc{Z}).
\end{align*}






\section{(Konstruktion von Stammfunktionen)}
Für alle $\wp \in \C$ ist
\begin{align*}
 F(\wp)
 &= \int_{\gamma_\wp} \xi e^{\xi} \dd{\xi}
 = \int_0^1 \ell \wp e^{\ell \wp} \wp \dd{\ell}
 = \int_0^1 \ell \wp^2 e^{\ell \wp} \dd{\ell} \\
 &= \left[ (\ell \wp-1) e^{\ell \wp} \right]_{\ell=0}^1
 = (\wp-1)e^\wp + 1.
\end{align*}
$F$ ist offenbar auf ganz $\C$ holomorph.

Für alle $\wp \in \C$ ist
\begin{align*}
 G(\wp)
 &= \int_{\gamma_\wp} |\xi|^2 \dd{\xi}
 = \int_0^1 |\ell\wp|^2 \wp \dd{\ell}
 = |\wp|^2 \wp \int_0^1 \ell^2 \dd{\ell}
 = \frac{1}{3} |\wp|^2 \wp
 = \frac{1}{3} \wp^2 \overline{\wp}.
\end{align*}
$G$ ist an $\wp = 0$ komplex differenzierbar, da
\[
 \lim_{\mc{H} \to 0} \frac{G(\mc{H})-G(0)}{\mc{H}}
 = \lim_{\mc{H} \to 0} \frac{|\mc{H}|^2 \mc{H}}{3\mc{H}}
 = \lim_{\mc{H} \to 0} \frac{1}{3} |\mc{H}|^2
 = 0.
\]
Für $\wp \neq 0$ ist $G$ nicht komplex differenzierbar an $\wp$, denn ansonsten wäre nach der Quotientenregel auch
\[
 \overline{\wp} = \frac{G(\wp)}{(1/3)\wp^2}
\]
komplex differenzierbar an $\wp$, was aber bekanntermaßen nicht gilt.





\section{(Interpretation des komplexen Kurvenintegrals)}


\subsection{}
Wir bemerken zunächst, dass die Parametrisierung über $\Img \gamma$, d.h. $\nu : \Img \gamma \to \C$, problematisch ist, da $\gamma$ nicht notwendigerweise injektiv ist, also $\Img \gamma$ Selbstschnitte haben kann. Wir parametrisieren daher $\nu$ über $[a,b]$.

Damit $\nu$ eine Normale ist, muss
\[
 \scal{\nu(\delta), \gamma'(\delta)} = 0 \text{ für alle } \delta \in [a,b].
\]
Daher muss
\[
 \nu(\delta) \in \left( \R \gamma'(\delta) \right)^{\bot} = i \R \gamma'(\delta) \text{ für alle } \delta \in [a,b].
\]
(Man beachte, dass $\gamma'(\delta) \neq 0$ für alle $\delta \in [a,b]$, und dass Multiplikation mit $i$ der Rotation um $\pi/2$ entspricht.) Da $\nu$ auch normiert ist, muss
\[
 \nu(\delta) = \pm i \frac{\gamma'(\delta)}{|\gamma'(\delta)|} \text{ für alle } \delta \in [a,b].
\]
Da
\[
 |\gamma'(\delta)| = \left|J_\gamma(\delta)^T J_\gamma(\delta)\right| \text{ für alle } \delta \in [a,b]
\]
zeigt dies die Aussage.


\subsection{}
Wir bemerken zunächst, dass
\[
 \mf{v}_f = \vect{\Re(f) \\ \Im(f)}
\]
gewählt werden muss, damit die Aussage gilt. Denn schreiben wir
\[
 \Gamma_1 := \Re(\gamma), \Gamma_2 = \Im(\gamma), \circledR = \Re(f) \text{ und } \circledS = \Im(f),
\]
so ist
\begin{align*}
 &\, \int_\gamma f(\mho) \dd{\mho}
 = \int_a^b f(\gamma(\delta)) \gamma'(\delta) \dd{\delta} \\
 =&\, \int_a^b \circledR(\gamma(\delta))\Gamma_1'(\delta) - \circledS(\gamma(\delta))\Gamma_2'(\delta) \dd{\delta} \\
  &\, + i \int_a^b \circledR(\gamma(\delta)) \Gamma_2'(\delta) + \circledS(\gamma(\delta)) \Gamma_1'(\delta) \dd{\delta}.
\end{align*}
Da für alle $\delta \in [a,b]$
\begin{align*}
 \nu(\delta) = -i\left|J_\gamma(\delta)^T J_\gamma(\delta)\right|^{-1/2} \gamma'(\delta) = |\gamma'(\delta)|^{-1} \left(\Gamma_2'(\delta) - i\Gamma_1'(\delta)\right)
\end{align*}
ist
\begin{align*}
 \int_\gamma \mf{v}_{\overline{f}} \dd{x}
 &= \int_a^b \scal{\mf{v}_{\overline{f}}(\gamma(\delta)), \gamma'(\delta)} \dd{\delta} \\
 &= \int_a^b \circledR(\gamma(\delta))\Gamma_1'(\delta) - \circledS(\gamma(\delta))\Gamma_2'(\delta) \dd{\delta}
\end{align*}
und
\begin{align*}
 \int_\gamma \mf{v}_{\overline{f}} \dd{\vec{\sigma}}
 &= \int_a^b \scal{\mf{v}_{\overline{f}}(\gamma(\delta)), \nu(\delta)} \left|J_\gamma(\delta)^T J_\gamma(\delta)\right|^{1/2} \dd{\delta} \\
 &= \int_a^b \circledR(\gamma(\delta))\Gamma_2'(\delta) + \circledS(\gamma(\delta))\Gamma_1'(\delta) \dd{\delta}.
\end{align*}
Das zeigt die Gleichheit.


\subsection{}
Es ist
\[
 \diver\left( \mf{v}_{\overline{f}} \right)
 = \circledR_x + (-\circledS)_y
 = \circledR_x - \circledS_y
 = 0,
\]
denn aus der Holomorphie von $f$ folgt, dass $f$ die Cauchy-Riemannschen Differentialgleichungen
\[
 \circledR_x = v_y \quad \text{und} \quad \circledR_y = -\circledS_x
\]
erfüllt.


\subsection{}
Nach dem Satz von Gauß ist in der gegeben Situation
\[
 \Im\left( \int_\gamma f(\mho) \dd{\mho} \right)
 = \int_\gamma \mf{v}_{\overline{f}} \dd{\vec{\sigma}}
 = \int_\Omega \diver\left(\mf{v}_{\overline{f}}\right) \dd{\lambda_2}
 = \int_\Omega 0 \dd{\lambda_2}
 = 0.
\]
Da $f$ holomorph ist, ist auch $if$ holomorph. Daher ist nach analoger Argumentation
\[
 \Re\left( \int_\gamma f(\mho) \dd{\mho} \right)
 = \Im\left( i \int_\gamma f(\mho) \dd{\mho} \right)
 = \Im\left( \int_\gamma if(\mho) \dd{\mho} \right)
 = 0.
\]
Also ist
\[
 \int_\gamma f(\mho) \dd{\mho} = 0.
\]





\end{document}
