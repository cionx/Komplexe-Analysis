\documentclass[a4paper,10pt]{article}
%\documentclass[a4paper,10pt]{scrartcl}

\usepackage{../mystyle}

\setromanfont[Mapping=tex-text]{Linux Libertine O}
% \setsansfont[Mapping=tex-text]{DejaVu Sans}
% \setmonofont[Mapping=tex-text]{DejaVu Sans Mono}

\title{\sc Einführung in die Komplexe Analysis \\ \Large Blatt 6}
\author{Jendrik Stelzner}
\date{\today}

\begin{document}
\maketitle





\section{(Konvergenzradius)}
Da
\[
 \sum_n a_n z^{2n}
 = \sum_n a_n \left(z^2\right)^n
\]
beträgt der Konvergenzradius der Reihe $\sum_n a_n z^{2n}$
\[
 \sqrt{R} \text{ falls } R < \infty \quad \text{und} \quad \infty \text{ falls } R = \infty.
\]
Für die Reihe $\sum_n a_n^2 z^n$ ergibt sich der Konvergenzradius
\[
 \frac{1}{\limsup_{n \to \infty} \sqrt[n]{|a_n^2|}}
 = \left( \frac{1}{\limsup_{n \to \infty} \sqrt{|a_n|}} \right)^2
 = R^2,
\]
wobei wir $\infty^2 = \infty$ verstehen. Kombiniert ergibt sich damit, dass $\sum_n a_n^2 z^{2n}$ einen Konvergenzradius von $R$ hat.

Da
\[
 0 < R = \frac{1}{\limsup_{n \to \infty} \sqrt[n]{|a_n|}}
\]
ist
\[
 \limsup_{n \to \infty} \sqrt[n]{|a_n|} < \infty.
\]
Deshalb ist
\[
 \frac{1}{\limsup_{n \to \infty} \sqrt[n]{|a_n|/n!}}
 = \frac{\lim_{n \to \infty} \sqrt[n]{n!}}{\limsup_{n \to \infty} \sqrt[n]{|a_n|}}
 = \infty.
\]
Der Konvergenzradius von $\sum_n (a_n/n!) z^n$ ist daher $\infty$.








\end{document}
