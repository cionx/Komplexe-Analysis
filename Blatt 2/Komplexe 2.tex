\documentclass[a4paper,10pt]{article}
%\documentclass[a4paper,10pt]{scrartcl}

\usepackage{../mystyle}

\setromanfont[Mapping=tex-text]{Linux Libertine O}
% \setsansfont[Mapping=tex-text]{DejaVu Sans}
% \setmonofont[Mapping=tex-text]{DejaVu Sans Mono}

\title{\sc Einführung in die Komplexe Analysis \\ \Large Blatt 2}
\author{Jendrik Stelzner}
\date{\today}

\begin{document}
\maketitle





\section{(Konjugierte Nullstellen)}
Bekanntermaßen handelt es sich bei der Konjugation um einen $\R$-Algebraauto\-morph\-ismus von $\C$ (dem einzigen neben der Identität $\id_\C$). Inbesondere ist $\bar{x} = x$ für alle $x \in \R$. Es ist daher für alle $\rho \in \C$
\[
 \overline{P(\rho)} = \overline{\sum_{k=0}^n a_k \rho^k} = \sum_{k=0}^n a_k \bar{\rho}^k = P(\bar{\rho}).
\]
Also ist für alle $\rho \in \C$
\[
 0 = P(\rho) \Leftrightarrow 0 = \overline{P(\rho)} \Leftrightarrow 0 = P(\bar{\rho}).
\]









\end{document}
