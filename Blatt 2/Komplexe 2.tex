\documentclass[a4paper,10pt]{article}
%\documentclass[a4paper,10pt]{scrartcl}

\usepackage{../mystyle}

\setromanfont[Mapping=tex-text]{Linux Libertine O}
% \setsansfont[Mapping=tex-text]{DejaVu Sans}
% \setmonofont[Mapping=tex-text]{DejaVu Sans Mono}

\title{\sc Einführung in die Komplexe Analysis \\ \Large Blatt 2}
\author{Jendrik Stelzner}
\date{\today}

\begin{document}
\maketitle





\section{(Konjugierte Nullstellen)}
Bekanntermaßen handelt es sich bei der Konjugation um einen $\R$-Algebraauto\-morph\-ismus von $\C$ (dem einzigen neben der Identität $\id_\C$). Inbesondere ist $\bar{x} = x$ für alle $x \in \R$. Es ist daher für alle $\rho \in \C$
\[
 \overline{P(\rho)} = \overline{\sum_{k=0}^n a_k \rho^k} = \sum_{k=0}^n a_k \bar{\rho}^k = P(\bar{\rho}).
\]
Also ist für alle $\rho \in \C$
\[
 0 = P(\rho) \Leftrightarrow 0 = \overline{P(\rho)} \Leftrightarrow 0 = P(\bar{\rho}).
\]





\addtocounter{section}{1}





\section{(Real- und Imaginärteil quadratischer Funktionen)}
Wir behaupten, dass $p(x,y) = ax^2 + bxy + cy^2$ mit $a,b,c \in \R$ genau dann Realteil des komplexen Polynoms $P(z) = Az^2 + Bz + C$ ist, wenn $c = -a$.

Ist $c = -a$, so ist für beliebiges $b \in \R$
\[
 p(x,y) = ax^2 + bxy - ay^2 = \Re\left(\left(a-\frac{1}{2}bi\right)(x+iy)^2\right).
\]

Sei andererseits $p = \Re(P)$. Da
\[
 \Re(C) = \Re(P(0)) = p(0,0) = 0
\]
ist $p = \Re(Az^2+Bz)$, wir können also o.B.d.A. davon ausgehen, dass $C=0$. Da für alle $z = x+iy \in \C$
\begin{align*}
 0 &= p(x,y) - p(-x,-y) = \Re(P(z))-\Re(P(-z)) \\
   &= \Re(P(z)-P(-z)) = \Re(Bz - B(-z)) = \Re(2Bz) = 2 \Re(Bz)
\end{align*}
ist $\Re(Bz) = 0$ für alle $z \in \C$. Da $0 = \Re(B \bar{B}) = \Re(|B|^2)= |B|^2$ folgt daraus, dass $B = 0$. Also ist $P(z) = Az^2$ für $A = x_A + iy_A \in \C$. Es ist daher
\[
 p(x,y) = \Re(Az^2) = \Re((x_A+iy_A)(x+iy)^2) = x_Ax^2 - 2y_Axy - x_A y^2,
\]
was die Behauptung zeigt.

Man bemerke noch, dass $p$ genau dann Imaginärteil eines komplexen Polynoms vom Grad $n$ ist, wenn $p$ Realteil eines komplexen Polynoms vom Grad $n$ ist, denn $p = \Im(P) \Leftrightarrow p = \Re(-iP)$, bzw. $p = \Re(P) \Leftrightarrow p = \Im(iP)$ für jedes komplexe Polynom $P$. Also ist $p$ genau dann Imaginärteil eines Polynoms $P(z) = Az^2+Bz+C$, wenn $a = -c$.












\end{document}
