\documentclass[a4paper,10pt]{article}
%\documentclass[a4paper,10pt]{scrartcl}

\usepackage{../mystyle}

\setromanfont[Mapping=tex-text]{Linux Libertine O}
% \setsansfont[Mapping=tex-text]{DejaVu Sans}
% \setmonofont[Mapping=tex-text]{DejaVu Sans Mono}

\title{Einführung in die Komplexe Analysis \\ \Large Blatt 7}
\author{Jendrik Stelzner}
\date{\today}

\begin{document}
\maketitle





\section{(Konvergenzverhalten von Potenzreihen)}
Wir können o.B.d.A. davon ausgehen, dass die Entwicklungsstelle der Potenzreihe bei $z_0 = 0$ liegt. Es sei $(a_n)_{n \in \N}$ die Folge der Koeffizienten der Potenzreihe, d.h.
\[
 f(z) = \sum_{n=0}^\infty a_n z^n \text{ für alle } z \in \C.
\]
Da diese Reihe auf ganz $\C$ gleichmäßig konvergiert gibt es ein $M \in \N, M \geq 1$, so dass
\[
 \left| \sum_{n=m}^\infty a_n z^n \right| < \frac{1}{2} \text{ für alle } z \in \C, m \geq M.
\]

Für alle $m \geq M$ ist daher für alle $z \in \C$
\begin{align*}
 \left| a_m z^m \right|
 &= \left| \sum_{n=m}^\infty a_n z^n - \sum_{n=m+1}^\infty a_n z^n \right| \\
 &\leq \left| \sum_{n=m}^\infty a_n z^n \right| + \left| \sum_{n=m+1}^\infty a_n z^n \right|
 < 1,
\end{align*}
also für alle $x \in \R, x > 0$
\[
 |a_m| < \frac{1}{x^m} \text{ für alle } m \geq M.
\]
Da $1/x^m \to 0$ für $x \to \infty$ (denn $m \geq M \geq 1$) folgt, dass
\[
 a_m = 0 \text{ für alle } m \geq M.
\]
Daher ist $f$ ein Polynom (dessen Grad höchstens $M-1$ ist).










\end{document}
