\documentclass[a4paper,10pt]{article}
%\documentclass[a4paper,10pt]{scrartcl}

\usepackage{../mystyle}

\setromanfont[Mapping=tex-text]{Linux Libertine O}
% \setsansfont[Mapping=tex-text]{DejaVu Sans}
% \setmonofont[Mapping=tex-text]{DejaVu Sans Mono}

\title{Einführung in die Komplexe Analysis \\ \Large Blatt 9}
\author{Jendrik Stelzner}
\date{\today}

\begin{document}
\maketitle




\section{(Stammgebiete und Stammfunktionen)}


\subsection{}
$\C$ ist ein konvexes Gebiet. Daher besitzt, wie auf dem letzten Zettel gezeigt, jede ganze Funktion eine Stammfunktion. Also ist $\C$ ein Stammgebiet.

$\C \setminus \{i\}$ ist kein Stammgebiet, denn
\[
 f : \C \setminus \{i\} \to \C^*, z \mapsto \frac{1}{z-i}
\]
ist holomorph auf $\C \setminus \{i\}$, aber da
\[
 \int_{\del D_1(i)} f(z) \dd{z}
 = \int_{\del D_1(i)} \frac{1}{z-i} \dd{z}
 = \int_{\del D_1(0)} \frac{1}{z} \dd{z}
 = 2 \pi i
 \neq 0
\]
besitzt $f$ auf $\C \setminus \{i\}$ keine Stammfunktion.


\subsection{}
Da $S_1$ und $S_2$ offen sind, ist auch $S_1 \cup S_2$ offen. Es sei $f : S_1 \cup S_2 \to \C$ holomorph. Da $S_1$ ein Stammgebiet ist, besitzt $f_{|S_1} : S_1 \to \C$ eine Stammfunktion $F_1 : S_1 \to \C$. Da $S_2$ ein Stammgebiet ist, besitzt $f_{|S_2} : S_2 \to \C$ eine Stammfunktion $F_2 : S_2 \to \C$. Da
\[
 F_{1 | S_1 \cap S_2}' = f_{|S_1 \cap S_2} = F_{2 | S_1 \cap S_2}'
\]
und $S_1 \cap S_2$ zusammenhängend ist, gibt es ein $c \in \C$ mit
\[
 F_1(z) = F_2(z) \text{ für alle } z \in S_1 \cap S_2.
\]
Es ist daher
\[
 F : S_1 \cup S_2 \to \C,
 z \mapsto \begin{cases} F_1(z) & \text{ falls } z \in S_1 \\ F_2(z) + c \text{ falls } z \in S_2 \end{cases}
\]
eine Stammfunktion von $f$ auf $S_1 \cup S_2$.

Es besitzt also jede auf $S_1 \cup S_2$ holomorphe Funkion dort auch eine Stammfunktion. Ist zusätzlich $S_1 \cap S_2 \neq \emptyset$, so ist $S_1 \cup S_2$ auch zusammenhängend und daher ein Stammgebiet.

Wir betrachten weiter die Gebiete
\begin{align*}
 R^+ &:= \{z \in \C \mid \Re(z) > 0\}, \\
 R^- &:= \{z \in \C \mid \Re(z) < 0\}, \\
 I^+ &:= \{z \in \C \mid \Im(z) > 0\} \text{ und } \\
 I^- &:= \{z \in \C \mid \Im(z) < 0\}.
\end{align*}
Diese sind konvex und somit Stammgebiete. Da
\begin{align*}
 R^+ \cap R^- &= \{z \in \C \mid \Re(z) > 0 \text{ oder } \Im(z) > 0\} \text{ und } \\
 I^+ \cap I^- &= \{z \in \C \mid \Re(z) < 0 \text{ oder } \Im(z) < 0\}
\end{align*}
nichtleer und zusammenhängend sind, sind $R^+ \cup I^+$ und $R^- \cup I^-$ Stammgebiete. Es ist jedoch
\begin{align*}
  &\, (R^+ \cup I^+) \cap (R^- \cap I^-) \\
 =&\,  \{z \in \C \mid \Re(z) > 0, \Im(z) < 0 \text{ oder } \Re(z) < 0, \Im(z) > 0\}
\end{align*}
nicht zusammenhängend und das Gebiet
\[
 (R^+ \cup I^+) \cup (R^- \cup I^-) = \C^*
\]
ist kein Stammgebiet. (Denn die Funktion
\[
 f : \C^* \to \C^*, z \mapsto \frac{1}{z}
\]
ist auf $\C^*$ holomorph, besitzt wegen
\[
 \int_{\del D_1(0)} f(z) \dd{z}
 = \int_{\del D_1(0)} \frac{1}{z} \dd{z}
 = 2 \pi i
\]
keine Stammfunktion auf $\C^*$.)















\end{document}
