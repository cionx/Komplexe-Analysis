\documentclass[a4paper,10pt]{article}
%\documentclass[a4paper,10pt]{scrartcl}

\usepackage{../mystyle}

\setromanfont[Mapping=tex-text]{Linux Libertine O}
% \setsansfont[Mapping=tex-text]{DejaVu Sans}
% \setmonofont[Mapping=tex-text]{DejaVu Sans Mono}

\title{Einführung in die Komplexe Analysis \\ \Large Blatt 9}
\author{Jendrik Stelzner}
\date{\today}

\begin{document}
\maketitle




\section{(Stammgebiete und Stammfunktionen)}


\subsection{}
$\C$ ist ein konvexes Gebiet. Daher besitzt, wie auf dem letzten Zettel gezeigt, jede ganze Funktion eine Stammfunktion. Also ist $\C$ ein Stammgebiet.

$\C \setminus \{i\}$ ist kein Stammgebiet, denn
\[
 f : \C \setminus \{i\} \to \C^*, z \mapsto \frac{1}{z-i}
\]
ist holomorph auf $\C \setminus \{i\}$, aber da
\[
 \int_{\del D_1(i)} f(z) \dd{z}
 = \int_{\del D_1(i)} \frac{1}{z-i} \dd{z}
 = \int_{\del D_1(0)} \frac{1}{z} \dd{z}
 = 2 \pi i
 \neq 0
\]
besitzt $f$ auf $\C \setminus \{i\}$ keine Stammfunktion.


\subsection{}
Da $S_1$ und $S_2$ offen sind, ist auch $S_1 \cup S_2$ offen. Es sei $f : S_1 \cup S_2 \to \C$ holomorph. Da $S_1$ ein Stammgebiet ist, besitzt $f_{|S_1} : S_1 \to \C$ eine Stammfunktion $F_1 : S_1 \to \C$. Da $S_2$ ein Stammgebiet ist, besitzt $f_{|S_2} : S_2 \to \C$ eine Stammfunktion $F_2 : S_2 \to \C$. Da
\[
 F_{1 | S_1 \cap S_2}' = f_{|S_1 \cap S_2} = F_{2 | S_1 \cap S_2}'
\]
und $S_1 \cap S_2$ zusammenhängend ist, gibt es ein $c \in \C$ mit
\[
 F_1(z) = F_2(z) \text{ für alle } z \in S_1 \cap S_2.
\]
Es ist daher
\[
 F : S_1 \cup S_2 \to \C,
 z \mapsto \begin{cases} F_1(z) & \text{ falls } z \in S_1 \\ F_2(z) + c \text{ falls } z \in S_2 \end{cases}
\]
eine Stammfunktion von $f$ auf $S_1 \cup S_2$.

Es besitzt also jede auf $S_1 \cup S_2$ holomorphe Funkion dort auch eine Stammfunktion. Ist zusätzlich $S_1 \cap S_2 \neq \emptyset$, so ist $S_1 \cup S_2$ auch zusammenhängend und daher ein Stammgebiet.

Wir betrachten weiter die Gebiete
\begin{align*}
 R^+ &:= \{z \in \C \mid \Re(z) > 0\}, \\
 R^- &:= \{z \in \C \mid \Re(z) < 0\}, \\
 I^+ &:= \{z \in \C \mid \Im(z) > 0\} \text{ und } \\
 I^- &:= \{z \in \C \mid \Im(z) < 0\}.
\end{align*}
Diese sind konvex und somit Stammgebiete. Da
\begin{align*}
 R^+ \cap R^- &= \{z \in \C \mid \Re(z) > 0 \text{ oder } \Im(z) > 0\} \text{ und } \\
 I^+ \cap I^- &= \{z \in \C \mid \Re(z) < 0 \text{ oder } \Im(z) < 0\}
\end{align*}
nichtleer und zusammenhängend sind, sind $R^+ \cup I^+$ und $R^- \cup I^-$ Stammgebiete. Es ist jedoch
\begin{align*}
  &\, (R^+ \cup I^+) \cap (R^- \cap I^-) \\
 =&\,  \{z \in \C \mid \Re(z) > 0, \Im(z) < 0 \text{ oder } \Re(z) < 0, \Im(z) > 0\}
\end{align*}
nicht zusammenhängend und das Gebiet
\[
 (R^+ \cup I^+) \cup (R^- \cup I^-) = \C^*
\]
ist kein Stammgebiet. (Denn die Funktion
\[
 f : \C^* \to \C^*, z \mapsto \frac{1}{z}
\]
ist auf $\C^*$ holomorph, besitzt wegen
\[
 \int_{\del D_1(0)} f(z) \dd{z}
 = \int_{\del D_1(0)} \frac{1}{z} \dd{z}
 = 2 \pi i
\]
keine Stammfunktion auf $\C^*$.)





\section{(Weierstraßscher Konvergenzsatz)}

\begin{lem}
 Sei $U \subseteq \C$ offen, $\alpha : [0,1] \to U$ eine stückweise stetig differenzierbare Kurve. Es sei $f_n : |\alpha| \to \C$ eine Folge stetiger Funktionen, die auf $|\alpha|$ lokal gleichmäßig gegen $f : |\alpha| \to \C$ konvergiert. Dann ist
 \[
  \int_\alpha f(z) \dd{z} = \lim_{n \to \infty} \int_\alpha f_n(z) \dd{z}.
 \]
\end{lem}\label{lem: Wegintegral und Grenzwert vertauschen}
\begin{proof}
 Wir betrachten zunächst den Fall, dass $\alpha$ stetig differenzierbar ist und $f_n$ auf $|\alpha|$ gleichmäßig gegen $f$ konvergiert. Da $\alpha$ stetig differenzierbar ist, gibt es wegen der Kompaktheit von $[0,1]$ ein $C > 0$ mit $|\alpha'(t)| < C$ für alle $t \in [0,1]$. Sei $\varepsilon > 0$ beliebig aber fest. Da $f_n$ auf $|\alpha|$ gleichmäßig gegen $f$ konvergiert, gibt es $N \in \N$ mit
 \[
  |f(z)-f_n(z)| \leq \frac{\varepsilon}{C} \text{ für alle } n \geq N, z \in |\alpha|.
 \]
 Daher ist
 \[
  |f(\alpha(t))\alpha'(t) - f_n(\alpha(t))\alpha'(t)| \leq \varepsilon \text{ für alle } n \geq N, t \in [0,1].
 \]
 Das zeigt, dass $(f_n \circ \alpha) \alpha'$ auf $[0,1]$ gleichmäßig gegen $(f \circ \alpha) \alpha'$ konvergiert. Daher ist
 \begin{align*}
  \int_\alpha f(z) \dd{z}
  &= \int_0^1 f(\alpha(t)) \alpha'(t) \dd{t}
  = \int_0^1 \lim_{n \to \infty} f_n(\alpha(t)) \alpha'(t) \dd{t} \\
  &= \lim_{n \to \infty} \int_0^1 f_n(\alpha(t)) \alpha'(t) \dd{t}
  = \lim_{n \to \infty} \int_\alpha f_n(z) \dd{z}.
 \end{align*}

Wir betrachten nun den Fall, dass $\alpha$ stetig differenzierbar ist, und $f_n$ auf $|\alpha|$ lokal gleichmäßig gegen $f$ konvergiert. Dann gibt es für jeden Punkt $z \in |\alpha|$ eine offene Umgebung $U_z \subseteq U$ von $z$, so dass $f_n$ auf $U_z \cap |\alpha|$ gleichmäßig gegen $f$ konvergiert. Wegen der Kompaktkeit von $|\alpha|$ hat die offene Überdeckung $\{U_z \mid z \in |\alpha|\}$ von $|\alpha|$ eine endliche Teilüberdeckung. Es gibt also $V_1, \ldots, V_n \in \{U_z \mid z \in |\alpha|\}$, so dass
\[
 |\alpha| \subseteq V_1 \cup \ldots \cup V_n.
\]
und $f_n$ für alle $k = 1, \ldots, n$ auf $V_k \cap |\alpha|$ gleichmäßig gegen $f$ konvergiert. Wegen der Endlichkeit dieser Überdeckung konvergiert $f_n$ auch auf
\[
 (V_1 \cap |\alpha|) \cup \ldots \cup (V_n \cap |\alpha|)
 = (V_1 \cup \ldots \cup V_n) \cap |\alpha|
 = |\alpha|
\]
gleichmäßig gegen $f$. Die Aussage ergibt sich daher aus dem vorherigen Fall.

Zuletzt betrachten wir den Fall, dass $\alpha$ stückweise stetig differenzierbar ist und $f_n$ auf $|\alpha|$ lokal gleichmäßig gegen $f$ konvergiert. Dann gibt es stetig differenzierbare Wege $\beta_1, \ldots, \beta_m : [0,1] \to \C$, so dass $\alpha = \beta_1 + \ldots + \beta_m$. Da $(f_n)$ auf $|\alpha|$ lokal gleichmäßig gegen $f$ konvergiert, konvergiert $(f_n)$ für alle $k = 1, \ldots, m$ auch auf $|\beta_k|$ lokal gleichmäßig gegen $f$. Daher ist nach dem vorherigen Fall
\begin{align*}
 \int_\alpha f(z) \dd{z}
 &= \sum_{k=1}^m \int_{\beta_k} f(z) \dd{z}
 = \sum_{k=1}^m \lim_{n \to \infty} \int_{\beta_k} f_n(z) \dd{z} \\
 &= \lim_{n \to \infty} \sum_{k=1}^m \int_{\beta_k} f_n(z) \dd{z}
 = \lim_{n \to \infty} \int_\alpha f_n(z) \dd{z}.
\end{align*}
\end{proof}

Sei $\Delta \subseteq U$ ein abgeschlossenen Dreieck. Da die $f_n$ alle holomorph sind ist nach dem Lemma von Goursat
\[
 \int_{\del \Delta} f_n(z) \dd{z} = 0 \text{ für alle } n \in \N.
\]
Daher ist nach Lemma \ref{lem: Wegintegral und Grenzwert vertauschen} auch
\[
 \int_{\del \Delta} f(z) \dd{z}
 = \lim_{n \to \infty} \int_{\del \Delta} f(z) \dd{z}
 = 0.
\]
Wegen der Beliebigkeit von $\Delta$ ist $f$ nach dem Satz von Morera holomorph.

















\end{document}
