\documentclass[a4paper,10pt]{article}
%\documentclass[a4paper,10pt]{scrartcl}

\usepackage{../mystyle}

\setromanfont[Mapping=tex-text]{Linux Libertine O}
% \setsansfont[Mapping=tex-text]{DejaVu Sans}
% \setmonofont[Mapping=tex-text]{DejaVu Sans Mono}

\title{\sc Einführung in die Komplexe Analysis \\ \Large Blatt 5}
\author{Jendrik Stelzner}
\date{\today}

\begin{document}
\maketitle





\section{(Kettenregel)}
Sehen wir $U \subseteq \R^2$ und $f : U \to \R^2$, so ist $f$ stetig differenzierbar. Setzen wir
\[
 u := \Re(f) = f_1 \text{ und } v := \Im(f) = f_2
\]
so ist nach der Kettenregel für alle $t \in (0,1)$
\begin{align*}
 D(f \circ \gamma)(t)
 &= Df(\gamma(t)) \cdot D\gamma(t) \\
 &= \vect{u_x(\gamma(t)) & u_y(\gamma(t)) \\ v_x(\gamma(t)) & v_y(\gamma(t))} \cdot \vect{\gamma_1'(t) \\ \gamma_2'(t)} \\
 &= \vect{u_x(\gamma(t)) \gamma_1'(t) + u_y(\gamma(t)) \gamma_2'(t) \\ v_x(\gamma(t)) \gamma_1'(t) + v_y(\gamma(t)) \gamma_2'(t)}.
\end{align*}
Schreiben wir diesen Ausdruck wieder als komplexe Zahl, und nutzen wir die Cauchy-Riemannschen Differentialgleichungen $u_x = v_y$ und $u_y = -v_x$, so erhalten wir, dass für alle $t \in (0,1)$
\begin{align*}
  &\; (f \circ \gamma)'(t) \\
 =&\; u_x(\gamma(t)) \gamma_1'(t) + u_y(\gamma(t))\gamma_2'(t) + i(v_x(\gamma(t)) \gamma_1'(t) + v_y(\gamma(t)) \gamma_2'(t)) \\
 =&\; u_x(\gamma(t)) \gamma_1'(t) - v_x(\gamma(t))\gamma_2'(t) + i(v_x(\gamma(t)) \gamma_1'(t) + u_x(\gamma(t)) \gamma_2'(t)) \\
 =&\; (u_x(\gamma(t))+iv_x(\gamma(t))) \cdot (\gamma_1'(t)+i\gamma_2'(t)) \\
 =&\; f'(\gamma(t)) \cdot \gamma'(t).
\end{align*}
Es gilt also überraschenderweise die Kettenregel.





\section{(Ausdehnung von Kurven)}
Wir gehen davon aus, dass $\gamma$ zweimal differenzierbar ist, und dass
\[
 \Psi, \Phi : (0,1) \times \R \to \R
\]
differenzierbar sind. Setzen wir
\[
 \gamma_1 := \Re(\gamma) \text{ und } \gamma_2 := \Im(\gamma),
\]
so ist
\begin{align*}
 u(x,y) &:= \Re(f(x,y)) = \gamma_1(x) + \Psi(x,y)\gamma_1'(x) - \Phi(x,y)\gamma_2'(x), \\
 v(x,y) &:= \Im(f(x,y)) = \gamma_2(x) + \Psi(x,y)\gamma_2'(x) + \Phi(x,y)\gamma_1'(x).
\end{align*}
Daher ist für alle $(x,y) \in (0,1) \times \R$
\begin{align*}
 u_x(x,y) =&\; \gamma_1'(x) + \Psi_x(x,y)\gamma_1'(x) + \Psi(x,y)\gamma_1''(x) \\
           &\; - \Phi_x(x,y)\gamma_2'(x) - \Phi(x,y)\gamma_2''(x), \\
 v_x(x,y) =&\; \gamma_2'(x) + \Psi_x(x,y)\gamma_2'(x) + \Psi(x,y)\gamma_2''(x) \\
           &\; + \Phi_x(x,y)\gamma_1'(x) + \Phi(x,y)\gamma_1''(x), \\
 u_y(x,y) =&\; \Psi_y(x,y)\gamma_1'(x) - \Phi_y(x,y)\gamma_2'(x), \\
 v_y(x,y) =&\; \Psi_y(x,y)\gamma_2'(x) + \Phi_y(x,y)\gamma_2'(x).
\end{align*}
Dass $f$ holomorph ist, ist äquivalent dazu, dass $f$ die Cauchy-Riemannschen Differentialgleichungen erfüllt, dass also $u_x = v_y$ und $u_y = -v_x$.

Für den Fall, dass $\gamma(t) = t^2$, bemerken wir, dass sich $\gamma$ zu $f : (0,1) \times \R \to \C$ mit $f(z) = z^2$ fortsetzen lässt. Wählen wir
\begin{align*}
 &\Psi(x,y) : (0,1) \times \R \to \R, (x,y) \mapsto -\frac{y^2}{2x} \text{ und}\\
 &\Phi(x,y) : (0,1) \times \R \to \R, (x,y) \mapsto y,
\end{align*}
so haben wir $\Psi(x,0) = \Psi_y(x,0) = \Phi(x,0) = 0$ für alle $x \in (0,1)$ und für alle $x+iy \in (0,1) \times \R$
\begin{align*}
 f(x+iy) &= (x+iy)^2 = x^2-y^2 + i2xy \\
         &= x^2 + \left(-\frac{y^2}{2x}\right) \cdot 2x + iy \cdot 2x \\
         &= \gamma(x) + \Psi(x,y)\gamma'(x) + i\Phi(x,y)\gamma'(x).
\end{align*}





\section{(Potenzreihenentwicklung)}
Für alle $z \in \C$ mit $|z| < 1$ ist bekanntermaßen
\[
 \frac{1}{1-z} = \sum_{k=0}^\infty z^k.
\]
Für $f : \C \setminus \{1,-1,i\} \to \C$ mit
\[
 f(z) := \frac{1}{z^3 - iz^2 - z + i}
\]
hat eine Potenzreihenentwicklung um die Entwicklungstelle $0$ einen Konvergenzradius von höchstens $1$, da $f$ bei $1,-1$ und $i$ Singularitäten aufweist. Für jedes $z \in \C$ mit $|z| < 1$ ist auch $|z^4| = |z|^4 < 1$, und somit
\begin{align*}
 f(z)
 &= \frac{1}{z^3 - iz^2 - z + i}
 = \frac{z-(-i)}{(z^3+(-i)z^2+(-i)^2z+(-i)^3)(z-(-i))} \\
 &= \frac{z+i}{z^4-1} = -(z+i)\frac{1}{1-z^4} = -(z+i) \sum_{k=0}^\infty \left(z^4\right)^k \\
 &= \sum_{k=0}^\infty -(z+i)z^{4k} = \sum_{k=0}^\infty -z^{4k+1}-iz^{4k} = \sum_{n=0}^\infty a_n z^n,
\end{align*}
mit
\[
 a_n =
 \begin{cases}
  -i & \text{falls } n \equiv 0 \bmod 4, \\
  -1 & \text{falls } n \equiv 1 \bmod 4, \\
   0 & \text{sonst}.
 \end{cases}
\]
Also lässt sich $f$ um $0$ mit einer Potenzreihe mit Konvergenzradius $1$ entwickeln.























\end{document}
