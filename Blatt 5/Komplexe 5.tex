\documentclass[a4paper,10pt]{article}
%\documentclass[a4paper,10pt]{scrartcl}

\usepackage{../mystyle}

\setromanfont[Mapping=tex-text]{Linux Libertine O}
% \setsansfont[Mapping=tex-text]{DejaVu Sans}
% \setmonofont[Mapping=tex-text]{DejaVu Sans Mono}

\title{\sc Einführung in die Komplexe Analysis \\ \Large Blatt 5}
\author{Jendrik Stelzner}
\date{\today}

\begin{document}
\maketitle





\section{(Kettenregel)}
Sehen wir $U \subseteq \R^2$ und $f : U \to \R^2$, so ist $f$ stetig differenzierbar. Setzen wir
\[
 u := \Re(f) = f_1 \text{ und } v := \Im(f) = f_2
\]
so ist nach der Kettenregel für alle $t \in (0,1)$
\begin{align*}
 D(f \circ \gamma)(t)
 &= Df(\gamma(t)) \cdot D\gamma(t) \\
 &= \vect{u_x(\gamma(t)) & u_y(\gamma(t)) \\ v_x(\gamma(t)) & v_y(\gamma(t))} \cdot \vect{\gamma_1'(t) \\ \gamma_2'(t)} \\
 &= \vect{u_x(\gamma(t)) \gamma_1'(t) + u_y(\gamma(t)) \gamma_2'(t) \\ v_x(\gamma(t)) \gamma_1'(t) + v_y(\gamma(t)) \gamma_2'(t)}.
\end{align*}
Schreiben wir diesen Ausdruck wieder als komplexe Zahl, und nutzen wir die Cauchy-Riemannschen Differentialgleichungen $u_x = v_y$ und $u_y = -v_x$, so erhalten wir, dass für alle $t \in (0,1)$
\begin{align*}
  &\; (f \circ \gamma)'(t) \\
 =&\; u_x(\gamma(t)) \gamma_1'(t) + u_y(\gamma(t))\gamma_2'(t) + i(v_x(\gamma(t)) \gamma_1'(t) + v_y(\gamma(t)) \gamma_2'(t)) \\
 =&\; u_x(\gamma(t)) \gamma_1'(t) - v_x(\gamma(t))\gamma_2'(t) + i(v_x(\gamma(t)) \gamma_1'(t) + u_x(\gamma(t)) \gamma_2'(t)) \\
 =&\; (u_x(\gamma(t))+iv_x(\gamma(t))) \cdot (\gamma_1'(t)+i\gamma_2'(t)) \\
 =&\; f'(\gamma(t)) \cdot \gamma'(t).
\end{align*}
Es gilt also überraschenderweise die Kettenregel.





\section{(Ausdehnung von Kurven)}
Wir gehen davon aus, dass $\gamma$ zweimal differenzierbar ist, und dass
\[
 \Psi, \Phi : (0,1) \times \R \to \R
\]
differenzierbar sind. Setzen wir
\[
 \gamma_1 := \Re(\gamma) \text{ und } \gamma_2 := \Im(\gamma),
\]
so ist
\begin{align*}
 u(x,y) &:= \Re(f(x,y)) = \gamma_1(x) + \Psi(x,y)\gamma_1'(x) - \Phi(x,y)\gamma_2'(x), \\
 v(x,y) &:= \Im(f(x,y)) = \gamma_2(x) + \Psi(x,y)\gamma_2'(x) + \Phi(x,y)\gamma_1'(x).
\end{align*}
Daher ist für alle $(x,y) \in (0,1) \times \R$
\begin{align*}
 u_x(x,y) =&\; \gamma_1'(x) + \Psi_x(x,y)\gamma_1'(x) + \Psi(x,y)\gamma_1''(x) \\
           &\; - \Phi_x(x,y)\gamma_2'(x) - \Phi(x,y)\gamma_2''(x), \\
 v_x(x,y) =&\; \gamma_2'(x) + \Psi_x(x,y)\gamma_2'(x) + \Psi(x,y)\gamma_2''(x) \\
           &\; + \Phi_x(x,y)\gamma_1'(x) + \Phi(x,y)\gamma_1''(x), \\
 u_y(x,y) =&\; \Psi_y(x,y)\gamma_1'(x) - \Phi_y(x,y)\gamma_2'(x), \\
 v_y(x,y) =&\; \Psi_y(x,y)\gamma_2'(x) + \Phi_y(x,y)\gamma_2'(x).
\end{align*}
Dass $f$ holomorph ist, ist äquivalent dazu, dass $f$ die Cauchy-Riemannschen Differentialgleichungen erfüllt, dass also $u_x = v_y$ und $u_y = -v_x$.

Für den Fall, dass $\gamma(t) = t^2$, bemerken wir, dass sich $\gamma$ zu $f : (0,1) \times \R \to \C$ mit $f(z) = z^2$ fortsetzen lässt. Wählen wir
\begin{align*}
 &\Psi(x,y) : (0,1) \times \R \to \R, (x,y) \mapsto -\frac{y^2}{2x} \text{ und}\\
 &\Phi(x,y) : (0,1) \times \R \to \R, (x,y) \mapsto y,
\end{align*}
so haben wir $\Psi(x,0) = \Psi_y(x,0) = \Phi(x,0) = 0$ für alle $x \in (0,1)$ und für alle $x+iy \in (0,1) \times \R$
\begin{align*}
 f(x+iy) &= (x+iy)^2 = x^2-y^2 + i2xy \\
         &= x^2 + \left(-\frac{y^2}{2x}\right) \cdot 2x + iy \cdot 2x \\
         &= \gamma(x) + \Psi(x,y)\gamma'(x) + i\Phi(x,y)\gamma'(x).
\end{align*}





\section{(Potenzreihenentwicklung)}
Für alle $z \in \C$ mit $|z| < 1$ ist bekanntermaßen
\[
 \frac{1}{1-z} = \sum_{k=0}^\infty z^k.
\]
Für $f : \C \setminus \{1,-1,i\} \to \C$ mit
\[
 f(z) := \frac{1}{z^3 - iz^2 - z + i}
\]
hat eine Potenzreihenentwicklung um die Entwicklungstelle $0$ einen Konvergenzradius von höchstens $1$, da $f$ bei $1,-1$ und $i$ Singularitäten aufweist. Für jedes $z \in \C$ mit $|z| < 1$ ist auch $|z^4| = |z|^4 < 1$, und somit
\begin{align*}
 f(z)
 &= \frac{1}{z^3 - iz^2 - z + i}
 = \frac{z-(-i)}{(z^3+(-i)z^2+(-i)^2z+(-i)^3)(z-(-i))} \\
 &= \frac{z+i}{z^4-1} = -(z+i)\frac{1}{1-z^4} = -(z+i) \sum_{k=0}^\infty \left(z^4\right)^k \\
 &= \sum_{k=0}^\infty -(z+i)z^{4k} = \sum_{k=0}^\infty -z^{4k+1}-iz^{4k} = \sum_{n=0}^\infty a_n z^n,
\end{align*}
mit
\[
 a_n =
 \begin{cases}
  -i & \text{falls } n \equiv 0 \bmod 4, \\
  -1 & \text{falls } n \equiv 1 \bmod 4, \\
   0 & \text{sonst}.
 \end{cases}
\]
Also lässt sich $f$ um $0$ mit einer Potenzreihe mit Konvergenzradius $1$ entwickeln.





\section{(Wirtingerkalkül)}


\subsection{}
Es ist klar, dass der Differentialoperator $\partd{}{x}$ auf $C^1(\Omega, \C)$ $\R$-linear ist. Da für alle $f \in C^1(\Omega, \C)$ und $z \in \Omega$
\[
 \partd{(if)}{x}(z) = {\lim_{h \to 0}}\frac{if(z+h)-if(z)}{h} = i\lim_{h \to 0} \frac{f(z+h)-f(z)}{h} = i\partd{f}{x}(z)
\]
ist $\partd{}{x}$ auch $\C$-linear. Analog ergibt sich, dass auch $\partd{}{y}$ $\C$-linear ist. Daher sind auch $\partd{}{z}$ und $\partd{}{\bar{z}}$ $\C$-linear.

Für den Differentialoperator $\partd{}{x}$ gilt die Produktregel, denn schreiben wir
\[
 u^f := \Re(f), v^f := \Im(f) \text{ und } u^g := \Re(g), v^g := \Im(g),
\]
so ist
\begin{align*}
 \partd{}{x}(fg)
 &= \partd{}{x}\left( u^f u^g - v^f v^g + i u^f v^g + i u^g v^f \right) \\
 &= u^f_x u^g + u^f u^g_x - v^f_x v^g - v^f v^g_x + i u^f_x v^g + i u^f v^g_x + i u^g_x v^f + i u^g v^f_x \\
 &= (u^f_x+iv^f_x)(u^g+iv^g) + (u^f+iv^f)(u^g_x+iv^g_x) \\
 &= \partd{f}{x}g + f\partd{g}{x}.
\end{align*}
Analog zeigt man auch, dass für $\partd{}{y}$ die Produktregel gilt. Da daher
\begin{align*}
 \partd{}{z}(fg)
 &= \frac{1}{2}\left( \partd{}{x}(fg) - i\partd{}{y}(fg) \right) \\
 &= \frac{1}{2}\left( \partd{f}{x}g + f\partd{g}{x} - i\partd{f}{y}g - if\partd{g}{y} \right) \\
 &= \frac{1}{2}\left( \partd{f}{x}-i\partd{f}{y} \right)g + \frac{1}{2}f\left( \partd{g}{x}-i\partd{g}{y} \right) \\
 &= \partd{f}{z}g + f\partd{g}{z}
\end{align*}
gilt auch für $\partd{}{z}$ die Produktregel. Da damit
\begin{align*}
 &\; \partd{f}{z} = \partd{}{z}\left(\frac{f}{g}g\right) = \partd{(f/g)}{z}g + \frac{f}{g}\partd{g}{z} \\
 \Rightarrow&\; \partd{f}{z}g = \partd{(f/g)}{z}g^2 + f\partd{g}{z} \\
 \Rightarrow&\; \partd{(f/g)}{z} = \frac{\partd{f}{z}g-f\partd{g}{z}}{g^2}.
\end{align*}
gilt für $\partd{}{z}$ auch die Quotientenregel. Analog zeigt man, dass $\partd{}{\bar{z}}$ die Quotienten- und Produktregel erfüllt.


\subsection{}
Schreiben wir $u := \Re(f)$ und $v := \Im(f)$, so folgt aus der $\C$-Linearität von $\partd{}{x}$ und $\partd{}{y}$, dass
\begin{equation}\label{eq: dfdz}
 \begin{aligned}
  \partd{f}{z}
  &= \frac{1}{2}\left( \partd{f}{x}-i\partd{f}{y} \right)
  = \frac{1}{2}\left( u_x+iv_x - i(u_y+iv_y) \right) \\
  &= \frac{1}{2}\left( u_x+v_y + i(v_x-u_y) \right)
 \end{aligned}
\end{equation}
und
\begin{equation}\label{eq: dfdkz}
 \begin{aligned}
  \partd{f}{\bar{z}}
  &= \frac{1}{2}\left( \partd{f}{x}+i\partd{f}{y} \right)
  = \frac{1}{2}\left( u_x+iv_x + i(u_y+iv_y) \right) \\
  &= \frac{1}{2}\left( u_x-v_y + i(v_x+u_y) \right).
 \end{aligned}
\end{equation}

Da $\Re\left(\bar{f}\right) = u$ und $\Im\left(\bar{f}\right) = -v$ ist daher
\[
 \overline{\left(\partd{f}{\bar{z}}\right)}
 = \frac{1}{2}\left( u_x-v_y - i(v_x+u_y) \right)
 = \partd{\bar{f}}{z}
\]
und
\[
 \overline{\left(\partd{f}{z}\right)}
 = \frac{1}{2}\left( u_x+v_y - i(v_x-u_y) \right)
 = \partd{\bar{f}}{\bar{z}}.
\]


\subsection{}
$f$ ist genau dann holomorph auf $\Omega$, wenn $f$ auf $\Omega$ die Cauchy-Riemannschen Differentialgleichungen erfüllt, also wenn
\[
 u_x = v_y \text{ und } u_y = -v_x.
\]
Dies ist nach \eqref{eq: dfdkz} offenbar äquivalent dazu, dass $\del f / \del \bar{z} = 0$. Ist $f$ holomorph auf $\Omega$, so gilt außerdem
\[
 f' = u_x + iv_x = v_y - iu_y = -i(u_y + iv_y),
\]
also $\del f / \del z = f'$.


\subsection{}
Da $\Re\left(\bar{f}\right) = u$ und $\Im\left(\bar{f}\right) = -v$ ist $\bar{f}$ nach den Cauchy-Riemannschen Differentialgleichungen genau dann holomorph auf $\Omega$, wenn
\[
 u_x = (-v)_y = -v_y \text{ und } u_y = -(-v_x) = v_x.
\]
Nach \eqref{eq: dfdz} ist dies äquivalent dazu, dass $\del f / \del z = 0$. Ist $\bar{f}$ holomorph, so gilt nicht, wie auf dem Zettel behautet, $\del \bar{f} / \del \bar{z} = \left(\bar{f}\right)'$, sondern $\del \bar{f} / \del z = \left(\bar{f}\right)'$, da
\[
 f' = u_x - i v_x = -v_y -v_x,
\]
und damit
\[
 \partd{\bar{f}}{z}
 = \frac{1}{2}\left( u_x-v_y - i(v_x+u_y) \right)
 = \left(\bar{f}\right)'.
\]


\subsection{}
Es ist
\[
 \det \vect{u_x & u_y \\ v_x & v_y} = u_x v_y - v_x u_y,
\]
sowie
\begin{align*}
 &\; \det \vect{\partd{f}{z} & \partd{f}{\bar{z}} \\ \partd{\bar{f}}{z} & \partd{\bar{f}}{\bar{z}}} \\
 =&\; \frac{1}{4}\left( u_x + v_y + iv_x - iu_y \right)\left( u_x + v_y - iv_x + iu_y \right) \\
  &\; -\frac{1}{4}\left( u_x - v_y + iv_x + iu_y \right)\left( u_x - v_y -iv_x - iu_y \right) \\
 =&\; \frac{1}{4}\left( (u_x+v_y)^2 - (iv_x-iu_y)^2 - (u_x-v_y)^2 + (iv_x+iu_y)^2 \right) \\
 =&\; \frac{1}{4}\left( 4 u_x v_y + 4 i^2 v_x u_y \right)
 = u_x v_y - v_x u_y
\end{align*}
und
\begin{align*}
 |f_z|^2 - |f_{\bar{z}}|^2
 &= \frac{1}{4} \left|x_x+v_y+i(v_x-u_y)\right|^2 - \frac{1}{4} \left|u_x-v_y+i(v_x+u_y)\right|^2 \\
 &= \frac{1}{4} \left( (u_x+v_y)^2 + (v_x-u_y)^2 - (u_x-v_y)^2 - (v_x+u_y)^2 \right) \\
 &= \frac{1}{4} \left( 4 u_x v_y - 4 v_x u_y \right)
 = u_x v_y - v_x u_y.
\end{align*}






















\end{document}
