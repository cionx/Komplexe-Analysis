\documentclass[a4paper,10pt]{article}
%\documentclass[a4paper,10pt]{scrartcl}

\usepackage{../mystyle}

\setromanfont[Mapping=tex-text]{Linux Libertine O}
% \setsansfont[Mapping=tex-text]{DejaVu Sans}
% \setmonofont[Mapping=tex-text]{DejaVu Sans Mono}

\title{\sc Einführung in die Komplexe Analysis \\ \Large Blatt 5}
\author{Jendrik Stelzner}
\date{\today}

\begin{document}
\maketitle





\section{(Kettenregel)}
Sehen wir $U \subseteq \R^2$ und $f : U \to \R^2$, so ist $f$ stetig differenzierbar. Setzen wir
\[
 u := \Re(f) = f_1 \text{ und } v := \Im(f) = f_2
\]
so ist nach der Kettenregel für alle $t \in (0,1)$
\begin{align*}
 D(f \circ \gamma)(t)
 &= Df(\gamma(t)) \cdot D\gamma(t) \\
 &= \vect{u_x(\gamma(t)) & u_y(\gamma(t)) \\ v_x(\gamma(t)) & v_y(\gamma(t))} \cdot \vect{\gamma_1'(t) \\ \gamma_2'(t)} \\
 &= \vect{u_x(\gamma(t)) \gamma_1'(t) + u_y(\gamma(t)) \gamma_2'(t) \\ v_x(\gamma(t)) \gamma_1'(t) + v_y(\gamma(t)) \gamma_2'(t)}.
\end{align*}
Schreiben wir diesen Ausdruck wieder als komplexe Zahl, und nutzen wir die Cauchy-Riemannschen Differentialgleichungen $u_x = v_y$ und $u_y = -v_x$, so erhalten wir, dass für alle $t \in (0,1)$
\begin{align*}
  &\; (f \circ \gamma)'(t) \\
 =&\; u_x(\gamma(t)) \gamma_1'(t) + u_y(\gamma(t))\gamma_2'(t) + i(v_x(\gamma(t)) \gamma_1'(t) + v_y(\gamma(t)) \gamma_2'(t)) \\
 =&\; u_x(\gamma(t)) \gamma_1'(t) - v_x(\gamma(t))\gamma_2'(t) + i(v_x(\gamma(t)) \gamma_1'(t) + u_x(\gamma(t)) \gamma_2'(t)) \\
 =&\; (u_x(\gamma(t))+iv_x(\gamma(t))) \cdot (\gamma_1'(t)+i\gamma_2'(t)) \\
 =&\; f'(\gamma(t)) \cdot \gamma'(t).
\end{align*}
Es gilt also überraschenderweise die Kettenregel.










\end{document}
